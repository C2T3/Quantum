\section{Multiplieur modulaire quantique de Pavlidis}
Pavlidis a proposé une manière de réaliser un multiplieur modulaire quantique, c'est-à-dire un circuit quantique qui calcule le résultat de $ax \text{ mod } N$ \cite{pavlidis2013fastquantummodularexponentiation}. Cependant, vu la demande en ressources que nécessite ce multiplieur modulaire autant classiquement que sur un ordinateur quantique, il s'avère complètement inutile pour l'utilisation qu'on veut en faire. La personne intéressée peut se référer à l'article pour en apprendre davantage, car cette méthode ne sera pas expliquée ici.