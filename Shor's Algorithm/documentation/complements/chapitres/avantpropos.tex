\section*{Avant-propros}
\addcontentsline{toc}{section}{Avant-propos}
Le présent document a été conçu lors de mon stage au C2T3 durant l'été 2024. Il s'agit d'un complément au document principal sur l'algorithme de Shor. Ainsi, on y trouve de la documentation sur les différents outils mathématiques qui permettent à l'algorithme de fonctionner, mais qui ne sont pas absolument nécessaires pour le réaliser. Encore une fois, je ne prétends pas être un expert des champs mathématiques que j'expose, car il s'agit là aussi de ma seule expérience jusqu'à maintenant.

Ce complément contient une section sur la transformée de Fourier, la théorie des groupes, la théorie des nombres, les fractions continues, l'explication du pseudocode et sur une manière de tricher pour rendre l'algorithme de Shor surpuissant. Finalement, les références se trouvent à la fin complètement.

J'espère que ce document pourra être utile. Bonne lecture!

\textit{- Mathis Beaudoin, étudiant au baccalauréat en sciences de l'information quantique à l'UdS}